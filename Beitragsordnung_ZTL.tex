\documentclass[a4paper, 12pt]{scrartcl}

%% PDF SETUP
\usepackage[pdftex, bookmarks, colorlinks, breaklinks,
pdfusetitle,plainpages=false]{hyperref}
\hypersetup{linkcolor=blue,pdfauthor={Zentrum für Technikkultur Landau},citecolor=blue,filecolor=black,urlcolor=blue,plainpages=false} 
\usepackage[utf8]{inputenc}
\usepackage[T1]{fontenc}
\usepackage{lmodern}
\usepackage[ngerman]{babel}
\usepackage{eurosym}

\usepackage{url}

% Optima as a sans serif font.
\renewcommand*\sfdefault{uop}
\usepackage[protrusion=true,expansion=true]{microtype}
% Recalculate page setup based on new font.
\KOMAoptions{DIV=last}
\usepackage{todonotes}
\pagestyle{plain}

\renewcommand*\thesection{\S~\arabic{section}}
\KOMAoptions{toc=flat}

\begin{document}
\title{Beitragsordnung}
\subtitle{des Vereins Zentrum für Technikkultur Landau e. V.}
\author{}
\date{}

\maketitle

\section{Höhe der Beiträge}\label{beitraege}
\begin{enumerate}
	\item Der reguläre Mitgliedsbeitrag beträgt 20 \euro/Monat.
	\item Der Mitgliedsbeitrag mit reduziertem Betrag für Schüler, Studenten und Auszubildende beträgt 10 \euro/Monat.\\ 
	Bei Antrag auf Mitgliedschaft muss der Status als Schüler, Student oder Auszubildender nachgewiesen werden.
	Eine Statusänderung muss dem Vorstand zeitnah mitgeteilt werden.
	\item Der Mitgliedsbeitrag für Familien beträgt 30 \euro/Monat und entspricht damit rechnerisch einer regulären und einer Mitgliedschaft mit reduziertem Beitrag. \\ 
Eine Familie besteht aus 2 Erwachsenen und bis zu 3 Kindern unter 18 Jahren, die alle die gleiche Postanschrift teilen.\\
Mit Vollenden des 18. Lebensjahrs ruht die Mitgliedschaft der Kinder bis der dann geltende Mitgliedsbeitrag bezahlt wird oder die Mitgliedschaft endet. 
	\item Der Mitgliedsbeitrag für Personen, die bereits im DARC e. V. Ortsverband Landau - K14 Mitglieder sind, beträgt 10 \euro/Monat und entspricht damit einer Mitgliedschaft mit reduziertem Beitrag.\\ Bei Antrag auf Mitgliedschaft muss die Mitgliedschaft im DARC e. V. Ortsverband Landau - K14 nachgewiesen werden.
	\item \label{beitraege-schulmitgliedschaft} Für Schulen und Bildungseinrichtungen wird eine Schulmitgliedschaft angeboten, diese beträgt 50 \euro/Monat. Sie ermöglicht zwei berechtigten Personen den Zutritt zum Space mit ihren Klassen/Schülern.
	\item Der Mitgliedsbeitrag für Fördermitglieder ist frei wählbar. \\
	Ab einem Beitrag von 50 \euro/Monat wird dem Fördermitglied eine Nennung auf der Vereins-Website ztl.space gewährt. \\
	Ab einem Beitrag von 100 \euro/Monat besteht zusätzlich die Möglichkeit, dass ein Schild an der Sponsorenwand im Vereinsheim installiert wird.
	\item Gemäß Satzung können in begründeten Einzelfällen individuelle Absprachen mit dem Vorstand getroffen werden. 
	\item Ehrenmitglieder sind laut Satzung beitragsfrei.
\end{enumerate}

\section{Fälligkeit}
\begin{enumerate}
	\item Der Mitgliedsbeitrag wird monatlich erhoben.
	\item Der Mitgliedsbeitrag wird bis zum dritten Werktag eines jeden Monats per Lastschriftverfahren eingezogen.
	\item Bei Zahlung des Beitrags per Überweisung (Dauerauftrag) ist diese bis zum dritten Werktag eines jeden Monats unaufgefordert zu tätigen.
	\item Bei Zahlung des Beitrags per Bargeld ist diese innerhalb eines jeden Monats unaufgefordert zu tätigen.
\end{enumerate}

\section{Zahlungsweise}
\begin{enumerate}
	\item Die Zahlung des Beitrags erfolgt per Lastschriftverfahren.
	\item Es stehen zwei Alternativen zur Zahlung des Beitrags per Lastschriftverfahren zur Verfügung:
	\begin{enumerate}
    	\item Die Zahlung des Beitrags per Überweisung (Dauerauftrag).
        \item Die Zahlung des Beitrags per Bargeld direkt an den Schatzmeister, sofern dieser zustimmt.
    \end{enumerate}
\end{enumerate}

\section{Aufnahmegebühr}
\begin{enumerate}
	\item Aufnahmegebühren werden nicht erhoben.
\end{enumerate}

\section{Erstattungen}
\begin{enumerate}
	\item Eine Erstattung von Mitgliedsbeiträgen findet nicht statt.
	\item Sollte dem Verein durch Zahlung des Beitrags per Lastschriftverfahren Kosten entstehen, die das Mitglied verursacht hat (z. B. keine ausreichende Kontodeckung, keine Änderungsmitteilung der Kontoverbindung, etc.) so sind diese vom Mitglied zu tragen und an den Verein zu erstatten.
\end{enumerate}

\section{Inkrafttreten}\label{inkrafttreten}
\begin{enumerate}
	\item Diese Beitragsordnung tritt am 11.09.2023 in Kraft.
\end{enumerate}


\vspace{2.5cm}

\noindent Geändert in \ref{beitraege}.\ref{beitraege-schulmitgliedschaft} (Höhe der Beiträge, neue Schulmitgliedschaft, Wegfall MEC-Mitgliedschaft) sowie \ref{inkrafttreten} in der Mitgliederversammlung am 09.09.2023. In Originalfassung bei der Mitgliederversammlung am 29.08.2020 beschlossen.

%\noindent Landau, 08.05.2019

\end{document}
